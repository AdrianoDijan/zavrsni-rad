\chapter{Zaključak}

Pomoću programske biblioteke i grafičkog programa, ovaj rad prikazuje
način implementacije i usporedbu triju algoritama za otkrivanje sudara
u trodimenzionalnom prostoru.
\\
Prvi dio rada opisuje tehnologije korištene
za realizaciju programske podrške, drugi dio prikazuje algoritme i optimizacije
koje su upotrebljavane, te je na kraju prikazana njihova primjena i rezultati.
\\
Iz rezultata testiranja vidljiv je značajan utjecaj oba algoritma optimizacije
u odnosu na referentni test metodom grube sile. Obje metode pokazale su poboljšanje
performansi do nekoliko redova veličine. Vidljivo je da metoda oktalnog stabla
u nekim slučajevima omogućava otkrivanje sudara objekata nakon samo jedne provjere,
dok je i u slučajevima kad sudar postoji do 1000 puta brža od metode grube sile.
\\
Kod metode prostornog hashiranja, iako u manjoj mjeri, također su vidljiva
značajna poboljšanja performansi. Kako prostorno hashiranje ovisi o većem broju
parametara (veličina čelije, hash funkcija), moguće je kako bi optimizacijom
tih parametara rezultati bili bolji.
