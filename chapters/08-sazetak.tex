\chapter*{Sažetak}
\addcontentsline{toc}{chapter}{Sažetak}

\section*{Sažetak}

Rad se bavi analizom dvaju algoritama optimizacije geometrijskih operacija,
metodom oktalnog stabla i prostornog hashiranja. U radu je prikazana
izrada C++ biblioteke za optimizaciju geometrijskih operacija i izrada
grafičke aplikacije za testiranje bibilioteke. Nakon uvoda u rad,
drugo poglavlje daje pregled korištenih tehnologija i načine na koje
su te tehnologije primjenjene u ovom radu. Sljedeća dva poglavlja
daju opis metoda optimizacije koje su korištene. Peto poglavlje prikazuje
načine i rezultate testiranja, te njihovu analizu. U šestom, posljednjem
poglavlju prije zaključka rada, prikazan je izgled i način funkcioniranja
grafičke aplikacije.

\section*{Ključne riječi}
\textit{
    Oktalno stablo, Prostorno hashiranje, Otkrivanje sudara, Qt3D
}


\chapter*{Summary}
\addcontentsline{toc}{chapter}{Summary}

\textbf{Title: } Optimization of geometric operations using the octal tree method and spatial hashing
\section*{Summary}

This thesis analyzes two algorithms used to optimize geometric operations,
the octal tree method, and the spatial hashing. The thesis demonstrated the process
of creating a C++ library used for the optimization of the geometric operations, 
and the GUI application used for testing the library. Following the introduction,
the second chapter gives a brief overview of the technologies and tools that were used
and the ways that those technologies were applied to this project.
The following two chapters describe the used optimization methods.
The fifth chapter shows the testing methods and provides the testing results
along with the analysis of those. The sixth chapter, the last one before the thesis
conclusion, shows the GUI application and explains the way it works.

\section*{Keywords}
\textit{
    Octree, Spatial hashing, Collision detection, Qt3D
}