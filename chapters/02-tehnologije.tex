\chapter{Alati i tehnologije}

\section{C++}

C++ je programski jezik opće namjene, izvorno napravljen od strane Bjarnea Stroustrupa.
Jezik je nastao kao proširenje programskog jezika C uz dodatak klasa. Moderne verzije jezika C++
podržavaju objektno orijentirano programiranje, generičko programiranje, funkcijski orjetirano programiranje.
Posljednja stabilna verzija standarda ovog programskog jezika je C++20, koji je korišten za izradu ovog rada.~\cite{cpp}

\section{Qt}

Qt je višeplatformsko okruženje za razvoj grafičkih sučelja i višeplatformskih GUI aplikacija.
Omogučava izradu grafičkih sučelja aplikacija za različite platforme uz korištenje standardnog API-ja.
Qt3D, jedna od bibilioteka Qt paketa, pruza mogučnost izrade 3D grafičkih aplikacija. Napisan je u jeziku C++.~\cite{qt}
Pri izradi rada, korišten je za izradu grafičkog sučelja i 3D prikaza objekata nad kojima se izvršava testiranje.

\section{CMake}

CMake je sustav otvorenog koda koji se koristi za automatizaciju izrade paketa, testiranje, prevođenje i instalaciju
softvera. Koristeći posebnu sintaksu (\textit{CMakeLists.txt}) generira konfiguracijske datoteke za druge programske prevoditelje
(MSVC, GCC, LLVM...) i razvojne okoline (Microsoft Visual Studio, Xcode...). Od verzije Qt6, podržani je način izrade Qt aplikacija. ~\cite{cmake}

\pagebreak
\section{Conan}

Conan je upravitelj paketa za C/C++ programske jezike. Omogućava preuzimanje biblioteka sa javno dostupnih repozitorija
i njihovu automatsku integraciju u programski kod. Moguće ga je integrirati sa CMake sustavom kako bi se postiglo automatsko
prevođenje i povezivanje potrebnih biblioteka sa izvornim kodom. Koristeći datoteku \textit{conanfile.py} moguće je definirati
skup pravila po kojima će se biblioteka ili programski paket prevoditi i pakirati, a takav paket je onda moguće koristiti lokalno
ili ga objaviti na javno dostupnom Conan repozitoriju.

\begin{pythonSource}{Primjer konfiguracijske datoteke (\textit{conanfile.py})}
from conans import ConanFile, CMake


class YAAACDConan(ConanFile):
    name = "yaaacd"
    version = "1.0"
    settings = "os", "compiler", "arch", "build_type"
    options = {"shared": [True, False]}
    default_options = "shared=False"
    generators = "cmake"
    exports_sources = (
        "src/*",
        "include/*",
        "CMakeLists.txt",
        "test/*"
    )

        cmake.definitions["CMAKE_BUILD_TYPE"] = str(
            self.settings.build_type
        ).upper()
        cmake.configure()
        return cmake

    def build(self):
        cmake = self._configure_cmake()
        cmake.build()
        cmake.test()

    def package(self):
        cmake = self._configure_cmake()
        cmake.install()

    def package_info(self):
        self.cpp_info.libs = ["yaaacd"]
\end{pythonSource}

\pagebreak

\section{Catch2}

Catch2 je bibilioteka korištena za testiranje C++ koda YAAACD biblioteke.
Omogućava jednostavno pisanje jediničnih testova uz korištenje dostupnih
\textit{macro}-a, automatsku registraciju testova i testove performansi.
Integrira se sa CMake paketom te ga je moguće koristiti sa CTest alatom
za testiranje.

\begin{cppSource}{Primjer Catch2 jediničnog testa}
TEST_CASE("Test octree collision", "[octree]") {
    std::vector<Triangle> triangles;

    std::vector<Vertex> vertices1{
        Vertex(0, 0, 0),
        Vertex(0, 0, 1),
        Vertex(0, 1, 0),
        Vertex(0, 1, 1),
        Vertex(1, 0, 0),
        Vertex(1, 0, 1),
        Vertex(1, 1, 0),
        Vertex(1, 1, 1),
    };

    std::vector<Vertex> vertices2;
    std::transform(
        vertices1.begin(),
        vertices1.end(),
        std::back_inserter(vertices2),
        [](const Vertex& vertex) {
            return Vertex(vertex.x + 0.5, vertex.y + 0.5, vertex.z + 0.5);
        }
    );

    std::vector<Triangle> cube1{
        {vertices1[0], vertices1[6], vertices1[4]},
        {vertices1[0], vertices1[2], vertices1[6]},
        {vertices1[0], vertices1[3], vertices1[2]},
        {vertices1[0], vertices1[1], vertices1[3]},
        {vertices1[2], vertices1[7], vertices1[6]},
        {vertices1[2], vertices1[3], vertices1[7]},
        {vertices1[4], vertices1[6], vertices1[7]},
        {vertices1[4], vertices1[7], vertices1[5]},
        {vertices1[0], vertices1[4], vertices1[5]},
        {vertices1[0], vertices1[5], vertices1[1]},
        {vertices1[1], vertices1[5], vertices1[7]},
        {vertices1[1], vertices1[7], vertices1[3]},
    };

    std::vector<Triangle> cube2{
        {vertices2[0], vertices2[6], vertices2[4]},
        {vertices2[0], vertices2[2], vertices2[6]},
        {vertices2[0], vertices2[3], vertices2[2]},
        {vertices2[0], vertices2[1], vertices2[3]},
        {vertices2[2], vertices2[7], vertices2[6]},
        {vertices2[2], vertices2[3], vertices2[7]},
        {vertices2[4], vertices2[6], vertices2[7]},
        {vertices2[4], vertices2[7], vertices2[5]},
        {vertices2[0], vertices2[4], vertices2[5]},
        {vertices2[0], vertices2[5], vertices2[1]},
        {vertices2[1], vertices2[5], vertices2[7]},
        {vertices2[1], vertices2[7], vertices2[3]},
    };

    Octree tree1(cube1);
    Octree tree2(cube2);

    REQUIRE(tree1.collides(&tree2));
}
\end{cppSource}