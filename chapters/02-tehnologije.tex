\chapter{Alati i tehnologije}

\section{C++}

C++ je programski jezik opće namjene, izvorno napravljen od strane Bjarnea Stroustrupa.
Jezik je nastao kao proširenje programskog jezika C uz dodatak klasa. Moderne verzije jezika C++
podržavaju objektno orijentirano programiranje, generičko programiranje, funkcijski orjetirano programiranje.
Posljednja stabilna verzija standarda ovog programskog jezika je C++20, koji je korišten za izradu ovog rada.~\cite{cpp}

\section{Qt}

Qt je višeplatformsko okruženje za razvoj grafičkih sučelja i višeplatformskih GUI aplikacija.
Omogučava izradu grafičkih sučelja aplikacija za različite platforme uz korištenje standardnog API-ja.
Qt3D, jedna od bibilioteka Qt paketa, pruza mogučnost izrade 3D grafičkih aplikacija. Napisan je u jeziku C++.~\cite{qt}
Pri izradi rada, korišten je za izradu grafičkog sučelja i 3D prikaza objekata nad kojima se izvršava testiranje.

\section{CMake}

CMake je sustav otvorenog koda koji se koristi za automatizaciju izrade paketa, testiranje, prevođenje i instalaciju
softvera. Koristeći posebnu sintaksu (\textit{CMakeLists.txt}) generira konfiguracijske datoteke za druge programske prevoditelje
(MSVC, GCC, LLVM...) i razvojne okoline (Microsoft Visual Studio, Xcode...). Od verzije Qt6, podržani je način izrade Qt aplikacija. ~\cite{cmake}

\pagebreak
\section{Conan}

Conan je upravitelj paketa za C/C++ programske jezike. Omogućava preuzimanje biblioteka sa javno dostupnih repozitorija
i njihovu automatsku integraciju u programski kod. Moguće ga je integrirati sa CMake sustavom kako bi se postiglo automatsko
prevođenje i povezivanje potrebnih biblioteka sa izvornim kodom. Koristeći datoteku \textit{conanfile.py} moguće je definirati
skup pravila po kojima će se biblioteka ili programski paket prevoditi i pakirati, a takav paket je onda moguće koristiti lokalno
ili ga objaviti na javno dostupnom Conan repozitoriju. \\

\textit{conanfile.py} datototeka je python izvršna datoteka koja koristi conans biblioteku za konfiguraciju Conan upravitelja paketa.
Informacije o paketu konfiguriramo kao varijable klase (u primjeru YAAACDConan) koja nasljeđuje klasu ConanFile.\\
Neki od konfiguracijskih parametara:
\begin{itemize}
  \item \textbf{name} - naziv paketa
  \item \textbf{version} - željena verzija paketa
  \item \textbf{settings} - parametri koji utječu na kompajliranje paketa\\
      Conan stvara hash vrijednosti koje se nalaze u \textit{settings} varijabli, te prema njima odlučuje treba li paket kompajlirati
      za neku platformu ili je moguće koristiti već kompajliranu verziju.
  \item \textbf{exports\_sources} - popis datoteka i mapa izvornog koda koje su potrebne za kompajliranje
\end{itemize} 

\pagebreak
Kako bi iz datoteka izvornog koda bilo moguće stvoriti biblioteku ili izvršnu datoteku, u datoteci \textit{conanfile.py} potrebno je definirati
funkcije koje su odgovorne za proces stvaranja paketa.
Potrebne funkcije:
\begin{itemize}
  \item \textbf{build} - izvršava se pri pozivu \textit{conan build} naredbe\\
      Konfigurira CMake alat i pokreće \textit{build} - proces kompajliranja i povezivanja.
  \item \textbf{package} - izvršava se pri pozivu \textit{conan package} naredbe\\
      Pakira datoteke izvornog koda u Conan paket. Ako projekt koristi CMake,
      događa se poziv ekvivalentan \textit{cmake --install} u Conan virtualnom okruženju.
\end{itemize}

U primjeru 2.1 definirana je i funkcija \textit{\_configure\_cmake} - pomoćna funkcija koja konfigurira radno okruženje i priprema ga za
kompajliranje paketa.


\begin{pythonSource}{Primjer konfiguracijske datoteke (\textit{conanfile.py})}
from conans import ConanFile, CMake


class YAAACDConan(ConanFile):
    name = "yaaacd"
    version = "1.0"

    settings = "os", "compiler", "arch", "build_type"
    options = {"shared": [True, False]}
    default_options = "shared=False"
    generators = "cmake"
    exports_sources = (
        "src/*",
        "include/*",
        "CMakeLists.txt",
        "test/*"
    )

    def _configure_cmake(self):
        cmake = CMake(self)
        cmake.definitions["CMAKE_BUILD_TYPE"] = str(
            self.settings.build_type
        ).upper()
        cmake.configure()
        return cmake

    def build(self):
        cmake = self._configure_cmake()
        cmake.build()
        cmake.test()

    def package(self):
        cmake = self._configure_cmake()
        cmake.install()

    def package_info(self):
        self.cpp_info.libs = ["yaaacd"]
\end{pythonSource}

\pagebreak

\section{Catch2}

Catch2 je bibilioteka korištena za testiranje C++ koda YAAACD biblioteke.
Omogućava jednostavno pisanje jediničnih testova uz korištenje dostupnih
\textit{macro}-a, automatsku registraciju testova i testove performansi.
Integrira se sa CMake paketom te ga je moguće koristiti sa CTest alatom
za testiranje.

Testovi se pišu u obliku C++ funkcija, s razlikom da definicija funkcije nema standardni oblik
\textit{<tip> naziv\_funkcije(argumenti...)}, već se koristi \textit{macro TEST\_CASE} kojem su argumenti
naziv testa i popis \textit{tagova} (Oznaka koje omogućavaju filtriranje testova kod kasnijeg izvršavanja, izrade izvještaja...). \\
Primjer 2.1 prikazuje test koji provjerava sudar među dvije \textit{Octree} strukture. Prvi dio funkcije je standardni C++ kod u kojem se
definiraju varijable potrebne za test, dok je odluka o uspješnosti testa definira koristeći \textit{REQUIRE macro}. \textit{REQUIRE macro}
funkcionira tako da testove smatra uspješnima ako je vrijednost argumenta istinita.
Uz \textit{REQUIRE}, dostupan je i \textit{REQUIRE\_FALSE} koji provjerava da je vrijednost u argumentu lažna.


\begin{cppSource}{Primjer Catch2 jediničnog testa}
TEST_CASE("Test octree collision", "[octree]") {
    std::vector<Triangle> triangles;

    std::vector<Vertex> vertices1{
        Vertex(0, 0, 0),
        Vertex(0, 0, 1),
        Vertex(0, 1, 0),
        Vertex(0, 1, 1),
        Vertex(1, 0, 0),
        Vertex(1, 0, 1),
        Vertex(1, 1, 0),
        Vertex(1, 1, 1),
    };

    std::vector<Vertex> vertices2;
    std::transform(
        vertices1.begin(),
        vertices1.end(),
        std::back_inserter(vertices2),
        [](const Vertex& vertex) {
            return Vertex(vertex.x + 0.5, vertex.y + 0.5, vertex.z + 0.5);
        }
    );

    std::vector<Triangle> cube1{
        {vertices1[0], vertices1[6], vertices1[4]},
        {vertices1[0], vertices1[2], vertices1[6]},
        {vertices1[0], vertices1[3], vertices1[2]},
        {vertices1[0], vertices1[1], vertices1[3]},
        {vertices1[2], vertices1[7], vertices1[6]},
        {vertices1[2], vertices1[3], vertices1[7]},
        {vertices1[4], vertices1[6], vertices1[7]},
        {vertices1[4], vertices1[7], vertices1[5]},
        {vertices1[0], vertices1[4], vertices1[5]},
        {vertices1[0], vertices1[5], vertices1[1]},
        {vertices1[1], vertices1[5], vertices1[7]},
        {vertices1[1], vertices1[7], vertices1[3]},
    };

    std::vector<Triangle> cube2{
        {vertices2[0], vertices2[6], vertices2[4]},
        {vertices2[0], vertices2[2], vertices2[6]},
        {vertices2[0], vertices2[3], vertices2[2]},
        {vertices2[0], vertices2[1], vertices2[3]},
        {vertices2[2], vertices2[7], vertices2[6]},
        {vertices2[2], vertices2[3], vertices2[7]},
        {vertices2[4], vertices2[6], vertices2[7]},
        {vertices2[4], vertices2[7], vertices2[5]},
        {vertices2[0], vertices2[4], vertices2[5]},
        {vertices2[0], vertices2[5], vertices2[1]},
        {vertices2[1], vertices2[5], vertices2[7]},
        {vertices2[1], vertices2[7], vertices2[3]},
    };

    Octree tree1(cube1);
    Octree tree2(cube2);

    REQUIRE(tree1.collides(&tree2));
}
\end{cppSource}


\section{Podešavanje radne okoline}

Projekt se sastoji od dva C++ paketa - biblioteke YAAACD i izvršnog programa YAAACD-GUI. Kako bi bilo moguće
kompajlirati i pokrenuti projekt, potrebno je konfigurirati radno okruženje za oba paketa.\\

Izvorni kod biblioteke nalazi se u GitHub repozitoriju \url{https://github.com/AdrianoDijan/yaaacd}.
Kako bi bilo moguće konfigurirati radnu okolinu, potrebno je imati instaliran Conan upravitelj paketa i CMake.
Conan se može preuzeti na \url{https://conan.io/downloads.html}, a CMake na \url{https://cmake.org/download/}.
Oba programa dostupna su za sve veće operacijske sustave (Linux, macOS, Windows), te se mogu preuzeti
i kroz sistemske upravitelje paketa (apt, pacman, brew, winget...).\\

Nakon što su oba paketa instalirana, potrebno je konfigurirati okolinu
za kompajliranje (\textit{engl. build environment}) na sljedeći način:

\begin{enumerate}
  \item U radnom direktoriju napraviti mapu \textit{build}
  \item Unutar mape \textit{build} pokrenuti naredbu \textit{conan install ..}
  \item Unutar iste mape pokrenuti \textit{conan package ..} te \textit{conan export ..}\\
      \textit{conan export ..} zapisat će paket u lokalnu predmemoriju kako bi ga \textit{YAAACD-GUI} mogao instalirati.
      Naredba \textit{conan package ..} pokrenut će kompajliranje biblioteke, a nakon uspješnog kompajliranja pokrenut će
      se i jedinični testovi.
\end{enumerate}

Nakon podešavanja i kompajliranja biblioteke YAAACD, potrebno je kompajlirati i GUI aplikaciju.
Izvorni kod potrebno je preuzeti iz GitHub repozitorija \url{https://github.com/AdrianoDijan/yaaacd-gui}.
U direktoriju u kojem je preuzet izvorni kod, potrebno je napraviti sljedeće korake:

\begin{enumerate}
  \item Prije kompajliranja potrebno je u Conan dodati Qt repozitorij, kako bi bilo moguće preuzeti Qt pakete.\\
      Kako bi se mogli koristiti Qt Conan repozitorijem, potrebno je registrirati se na \url{https://my.qt.io},
      te prema uputama sa \url{https://wiki.qt.io/Using_Conan_for_Qt6#Connecting_to_Conan_remote}, konfigurirati
      Conan da koristi taj repozitorij.
  \item U radnom direktoriju napraviti mapu \textit{build}
  \item Unutar mape \textit{build} pokrenuti naredbu \textit{cmake ..}
  \item Unutar iste mape pokrenuti \textit{cmake --build . (--parallel)} - opcionalno, ako 
    želimo da se za kompajliranje koristi više procesorskih niti
\end{enumerate}

Nakon kompajliranja, izvršna datoteka naziva \textit{yaaacd-gui} bit će dostupna u mapi \textit{build}.
Program prima 2 argumenta - nazive datoteka koje sadrže modele koje želimo testirati za sudare.
